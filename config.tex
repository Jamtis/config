%% Geometry
\ifbeamer{
    % \setbeameroption{show notes on second screen = right}
}{
    % Geometry
    \geometry{
        a4paper,
        textwidth=15cm,
        textheight=24cm,
        heightrounded,
        hratio=1:1,
        vratio=2:3,
    }
    
    \ifllncspaper{
        % We need page numbers according to https://crypto.iacr.org/2021/callforpapers.html
        \pagestyle{plain}
    }{
        \newtheoremstyle{break} % name
          {}                    % Space above, empty = `usual value'
          {}                    % Space below
          {}                    % Body font
          {}                    % Indent amount (empty = no indent, \parindent = para indent)
          {\bfseries}           % Thm head font
          {}                    % Punctuation after thm head
          {\newline}            % Space after thm head: \newline = linebreak
          {}                    % Thm head spec
        
        \theoremstyle{break}
        \newtheorem{theorem}{Theorem}[section]
        \newtheorem{remark}[theorem]{Remark}
        \newtheorem{corollary}[theorem]{Corollary}
        \newtheorem{lemma}[theorem]{Lemma}
        \newtheorem{definition}[theorem]{Definition}
        \newtheorem{construction}[theorem]{Construction}
        \newtheorem{example}{Example}[theorem]
        \newtheorem{observation}{Observation}
        %\newtheorem{notation}[theorem]{Notation}
        %\crefname{observation}{observation}{observations}
        \Crefname{observation}{Observation}{Observations}
    }
    \declaretheorem[name=Notation]{notation}
    \declaretheorem[name=Observation]{observation}
    \let\claim\relax
    \declaretheorem[name=Claim]{claim}
    \crefname{claim}{Claim}{Claims}
}


%% Todos
\ifllncspaper{
    \ifoptiondraft{% Larger margins to have todos readable
        \paperwidth=\dimexpr \paperwidth + 10cm\relax
        \oddsidemargin=\dimexpr\oddsidemargin + 5cm\relax
        \evensidemargin=\dimexpr\evensidemargin + 5cm\relax
        \marginparwidth=\dimexpr \marginparwidth + 5cm\relax
    }{}
    \ifluatex \else
        \renewcommand{\marginpar}{\marginnote}
    \fi
}{
    \presetkeys{todonotes}{inline}{}
}


%% environments
\newenvironment{nalign}{
    \begin{equation}
        \begin{aligned}
}{
        \end{aligned}
    \end{equation}
    \ignorespacesafterend
}

\newenvironment{sitemize}{
    \begin{itemize}[leftmargin=3.5mm]
}{
    \end{itemize}
    \ignorespacesafterend
}

\makeatletter
\let\oldframetitle\mdf@@frametitle@use
\let\mdf@@frametitle@use\relax
\providecommand{\mdf@@frametitle@use}{}
\makeatother
\newcommand{\newcontdbox}[2]{
    \newenvironment{#1}[1]{
        \providecommand{\sitem}{}
        % \tolerance=6500
        \tolerance=99999
        \begin{mdframed}[
            frametitle={#2 ##1 (cont'd)},
            repeatframetitle=true,
            frametitlealignment=\center,
            frametitlebelowskip=0,
            frametitlefont=\textrm,
            innertopmargin=2mm,
            innerleftmargin=1mm,
            innerrightmargin=1mm,
            everyline=true,
            needspace=6cm
        ]
            \mdfsubtitle[
                subtitleaboveskip=0,
                subtitlebelowskip=0,
                subtitleinneraboveskip=0,
                subtitleinnerbelowskip=1mm,
                subtitlefont=\textrm
            ]{#2 ##1}
    }{
        \end{mdframed}
        \ignorespacesafterend
    }
}
\newcontdbox{functionalitybox}{Functionality}
\newcontdbox{protocolbox}{Protocol}
\newcontdbox{simulatorbox}{Simulator for}

% https://tex.stackexchange.com/questions/391726/the-quotation-environment
\NewDocumentCommand{\bywhom}{m}{% the Bourbaki trick
  {\nobreak\hfill\penalty50\hskip1em\null\nobreak
   \hfill\mbox{\normalfont{#1}}%
   \parfillskip=0pt \finalhyphendemerits=0 \par}%
}

\NewDocumentEnvironment{pquotation}{m}{
    \begin{quoting}[
        leftmargin=\parindent,
        rightmargin=\parindent,]\itshape
}{
    \end{quoting}
    \vspace{-3mm}
    \bywhom{#1}
}

\NewDocumentEnvironment{hybrids}{}{
    \begin{enumerate}[label=Hybrid \(\mathcal{H}_{\arabic*}\)]
}{
    \end{enumerate}%
}




%% lists
\ifbeamer{}{
    \renewcommand\labelitemi{$\vcenter{\hbox{\tiny$\bullet$}}$}
    \setlist{noitemsep, topsep=0pt}
}


%% tikz
\usetikzlibrary{calc}
\usetikzlibrary{patterns}
\usetikzlibrary{shapes.misc}
\usetikzlibrary{shapes.callouts}
\tikzset{
    square/.style={rectangle, minimum width=.8cm, minimum height=.8cm, draw=black!60},
    % party/.style={rectangle, minimum width=.8cm, minimum height=.8cm, draw=black!60, rounded corners=.2cm},
    party/.style={circle, minimum size=.8cm, draw=black!60},
    func/.style={rectangle, draw=black, minimum width = .8cm, minimum height = .8cm},
    title/.style={rectangle, minimum width=.8cm, minimum height=.8cm},
    label/.style={rectangle, minimum width=.8cm, minimum height=.8cm},
    dummyicon/.style={circle, inner sep=0.05cm, minimum size=.8cm, draw=black!60, dotted, line width=.03cm},
    partyset/.style={circle, inner sep=0.05cm, minimum size=.8cm, draw=black!60, densely dashed, line width=.03cm},
    partyicon/.style={circle, inner sep=0.05cm, minimum size=.8cm, draw=black!60, fill=white},
    placeholder/.style={circle, inner sep=0.05cm, minimum size=.8cm},
    advicon/.style={circle, inner sep=0.05cm, minimum size=.8cm, draw=black!60, double},
    smallpartyicon/.style={circle, inner sep=0.05cm, minimum size=.08cm, draw=black!60},
    line/.style={densely dotted, line width=.02cm},
    functionalitybox/.style={draw=black!70,fill=black!70, text=white, minimum width=1cm,minimum height=.6cm}
}


%% Print URLs not in Typewriter Font
\def\UrlFont{\rm}


%% GLS
\newlength\glsnamewidth
\newlength\glssymbolwidth
\settowidth{\glsnamewidth}{\textbf{Name ................}}
\settowidth{\glssymbolwidth}{\textbf{Symbol}}
\newglossarystyle{symbolnamedesc}{%
  \setlength{\glsdescwidth}{\linewidth-\glsnamewidth-\glssymbolwidth-6\tabcolsep}%
  \renewenvironment{theglossary}%
    {\begin{longtable}{p{\glssymbolwidth}p{\glsnamewidth}p{\glsdescwidth}}}%
    {\end{longtable}}%
 \renewcommand*{\glossaryheader}{%
    \bfseries Symbol
   & \bfseries Name
   & \bfseries Description
   \tabularnewline
   \midrule
   \tabularnewline\endhead}%
  \renewcommand*{\glsgroupheading}[1]{}%
  \renewcommand{\glossentry}[2]{%
    \glossentrysymbol{##1} &
    \glstarget{##1}{\glossentryname{##1}} &
    \glossentrydesc{##1}\tabularnewline
  }%
  \renewcommand{\subglossentry}[3]{\glossentry{##2}{##3}}%
  \renewcommand*{\glsgroupskip}{}%
}

\newglossarystyle{acrodesc}{%
  \setlength{\glsdescwidth}{\linewidth-\glsnamewidth-3\tabcolsep}%
  \renewenvironment{theglossary}%
    {\begin{longtable}{p{\glsnamewidth}p{\glsdescwidth}}}%
    {\end{longtable}}%
 \renewcommand*{\glossaryheader}{%
     \bfseries Acronym
   & \bfseries Description
   \tabularnewline
   \midrule
   \tabularnewline\endhead}%
  \renewcommand*{\glsgroupheading}[1]{}%
  \renewcommand{\glossentry}[2]{%
    \glstarget{##1}{\glossentryname{##1}} &
    \glossentrydesc{##1}\tabularnewline
  }%
  \renewcommand{\subglossentry}[3]{\glossentry{##2}{##3}}%
  \renewcommand*{\glsgroupskip}{}%
}
\glsdisablehyper


%% no \newpage for includes
\ifbeamer{}{
	\let\originalinclude\include
	\renewcommand*{\include}[1]{
        \begingroup
		\let\clearpage\relax
		\originalinclude{#1}
        \endgroup
	}
}

%% square at end of each proof
\let\doendproof\endproof
\renewcommand\endproof{~\hfill\qed\doendproof}


%% ulem...
\normalem


%% Equations alignment
\AtBeginDocument{\setlength\abovedisplayskip{4pt}}
\AtBeginDocument{\setlength\belowdisplayskip{4pt}}


%% Typography
\renewcommand*{\bibfont}{\small}
\ifbeamer{
    \AtBeginBibliography{\fontsize{3}{3}}
}{}


%% Table
\def\arraystretch{1.2}
\AtBeginDocument{\setlength{\tabcolsep}{0.5em}}


%% save compile time
\ifdraft{
    % this may cause some reference to to be resolved correctly
    \let\import\include
}{
    \let\import\input
}

%% default hyper setup
\ifsubmission
    \hypersetup{
        linktoc=all,
        pdfauthor={},
        pdftitle={Anonymous submission},
        pdfsubject={},
        pdfkeywords={}
    }
\fi
\setcounter{tocdepth}{2}


%% hyper-xr
\makeatletter
\newcommand*{\addFileDependency}[1]{% argument=file name and extension
  \typeout{(#1)}
  \@addtofilelist{#1}
  \IfFileExists{#1}{}{\typeout{No file #1.}}
}
\makeatother

\newcommand*{\myexternaldocument}[1]{%
    \externaldocument{#1}%
    \addFileDependency{#1.tex}%
    \addFileDependency{#1.aux}%
}


%% my TOC
\newcommand{\mytoc}{
    \begingroup
        \let\clearpage\relax
        \let\chapter\section
        \def\contentsname{Contents}
        \tableofcontents
    \endgroup
}


%% my maketitle
\newcommand{\mymaketitle}{
    \begingroup
        \let\oldaddcontentsline\addcontentsline
        \renewcommand*{\addcontentsline}[3]{%
            \ifstrequal{##1}{toc}{%
                \ifstrequal{##2}{author}{}{%
                    \ifstrequal{##2}{title}{}{%
                        \oldaddcontentsline{##1}{##2}{##3}
                    }
                }
            }{%
                \oldaddcontentsline{##1}{##2}{##3}
            }
        }
        %\let\addtocontents\relax
        \maketitle
    \endgroup
}


%% Typography
\renewcommand*{\bibfont}{\small}