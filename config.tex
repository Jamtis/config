% https://tex.stackexchange.com/questions/128797/a-renewcommand-that-does-not-care-if-a-command-is-already-defined
\makeatletter
\newcommand*{\declarecommand}{%
  \@star@or@long\declare@command
}
\newcommand*{\declare@command}[1]{%
  \provide@command{#1}{}%
  % \let#1\@empty % would be more efficient, but without error checking
  \renew@command{#1}%
}
\makeatother


%% Geometry
\ifbeamer{
    % \setbeameroption{show notes on second screen = right}
}{
    \ifdraft{
        \geometry{
            paperwidth=25cm,
            textheight=24cm,
            heightrounded,
            left=1cm,
            right=8cm,
            marginparwidth=7cm
        }
    }{
        \ifeprint
            \geometry{
                a4paper,
                left=1in,
                right=1in,
            }
        \fi
    }
}


%% Environments
\ifbeamer{}{
    \ifllncspaper{}{
        % environments
        \newtheoremstyle{break} % name
          {}                    % Space above, empty = `usual value'
          {}                    % Space below
          {}                    % Body font
          {}                    % Indent amount (empty = no indent, \parindent = para indent)
          {\bfseries}           % Thm head font
          {}                    % Punctuation after thm head
          {\newline}            % Space after thm head: \newline = linebreak
          {}                    % Thm head spec
        
        \theoremstyle{break}
        \newtheorem{theorem}{Theorem}[section]
        \newtheorem{remark}[theorem]{Remark}
        \newtheorem{corollary}[theorem]{Corollary}
        \newtheorem{lemma}[theorem]{Lemma}
        \newtheorem{definition}[theorem]{Definition}
        \newtheorem{construction}[theorem]{Construction}
        \newtheorem{example}{Example}[theorem]
        %\newtheorem{observation}{Observation}
        %\newtheorem{notation}[theorem]{Notation}
        %\crefname{observation}{observation}{observations}
        %\Crefname{observation}{Observation}{Observations}
    }
    \declaretheorem[name=Notation]{notation}
    \declaretheorem[name=Observation]{observation}
    \let\claim\relax
    \declaretheorem[name=Claim]{claim}
    \crefname{claim}{Claim}{Claims}
}
\newenvironment{nalign}{
    \begin{equation}
        \begin{aligned}
}{
        \end{aligned}
    \end{equation}
    \ignorespacesafterend
}

\newenvironment{sitemize}{
    \begin{itemize}[leftmargin=3.5mm]
}{
    \end{itemize}
    \ignorespacesafterend
}

\makeatletter
\let\oldframetitle\mdf@@frametitle@use
\let\mdf@@frametitle@use\relax
\declarecommand{\mdf@@frametitle@use}{}
\makeatother
\newcommand{\newcontdbox}[2]{
    \newenvironment{#1}[1]{
        \declarecommand{\sitem}{}
        % \tolerance=6500
        \tolerance=99999
        \begin{mdframed}[
            frametitle={#2 ##1 (cont'd)},
            repeatframetitle=true,
            frametitlealignment=\center,
            frametitlebelowskip=0,
            frametitlefont=\textrm,
            innertopmargin=2mm,
            innerleftmargin=1mm,
            innerrightmargin=1mm,
            everyline=true,
            needspace=6cm
        ]
            \mdfsubtitle[
                subtitleaboveskip=0,
                subtitlebelowskip=0,
                subtitleinneraboveskip=0,
                subtitleinnerbelowskip=1mm,
                subtitlefont=\textrm
            ]{#2 ##1}
    }{
        \end{mdframed}
        \ignorespacesafterend
    }
}
\newcontdbox{functionalitybox}{Functionality}
\newcontdbox{protocolbox}{Protocol}
\newcontdbox{simulatorbox}{Simulator for}

% https://tex.stackexchange.com/questions/391726/the-quotation-environment
\NewDocumentCommand{\bywhom}{m}{% the Bourbaki trick
  {\nobreak\hfill\penalty50\hskip1em\null\nobreak
   \hfill\mbox{\normalfont{#1}}%
   \parfillskip=0pt \finalhyphendemerits=0 \par}%
}

\NewDocumentEnvironment{pquotation}{m}{
    \begin{quoting}[
        leftmargin=\parindent,
        rightmargin=\parindent,]\itshape
}{
    \end{quoting}
    \vspace{-3mm}
    \bywhom{#1}
}

\NewDocumentEnvironment{hybrids}{}{
    \begin{enumerate}[label=\(\mathcal{H}_{\arabic*}\)]
}{
    \end{enumerate}%
}

% https://tex.stackexchange.com/questions/223948/suggestions-for-nested-proofs
\newenvironment{subproof}[1][\proofname]{%
  \renewcommand{\qed}{\(\blacksquare\)}%
  \begin{proof}[#1]%
}{%
  \end{proof}%
}


%% Square at end of each proof
\let\originalendproof\endproof
\renewcommand\endproof{~\hfill\originalendproof}


%% Todos
\providecommand{\marginnote}[1]{}
\ifllncspaper{
    \ifluatex\else
        \renewcommand{\marginpar}{\marginnote}
    \fi
}{
    \presetkeys{todonotes}{inline}{}
}



%% lists
\ifbeamer{}{
    \renewcommand\labelitemi{$\vcenter{\hbox{\tiny$\bullet$}}$}
    \setlist{noitemsep, topsep=0pt}
}


%% tikz
\usetikzlibrary{calc}
\usetikzlibrary{patterns}
\usetikzlibrary{shapes.misc}
\usetikzlibrary{shapes.callouts}
\tikzset{
    square/.style={rectangle, minimum width=.8cm, minimum height=.8cm, draw=black!60},
    % party/.style={rectangle, minimum width=.8cm, minimum height=.8cm, draw=black!60, rounded corners=.2cm},
    party/.style={circle, minimum size=.8cm, draw=black!60},
    func/.style={rectangle, draw=black, minimum width = .8cm, minimum height = .8cm},
    title/.style={rectangle, minimum width=.8cm, minimum height=.8cm},
    label/.style={rectangle, minimum width=.8cm, minimum height=.8cm},
    dummyicon/.style={circle, inner sep=0.05cm, minimum size=.8cm, draw=black!60, dotted, line width=.03cm},
    partyset/.style={circle, inner sep=0.05cm, minimum size=.8cm, draw=black!60, densely dashed, line width=.03cm},
    partyicon/.style={circle, inner sep=0.05cm, minimum size=.8cm, draw=black!60, fill=white},
    placeholder/.style={circle, inner sep=0.05cm, minimum size=.8cm},
    advicon/.style={circle, inner sep=0.05cm, minimum size=.8cm, draw=black!60, double},
    smallpartyicon/.style={circle, inner sep=0.05cm, minimum size=.08cm, draw=black!60},
    line/.style={densely dotted, line width=.02cm},
    functionalitybox/.style={draw=black!70,fill=black!70, text=white, minimum width=1cm,minimum height=.6cm}
}


%% Print URLs not in Typewriter Font
\def\UrlFont{\rm}


%% Page numbering
\ifllncspaper{
    % We need page numbers according to https://crypto.iacr.org/2021/callforpapers.html
    \pagestyle{plain}
}{}


%% ulem...
\normalem


%% Equations alignment
\AtBeginDocument{\setlength\abovedisplayskip{4pt}}
\AtBeginDocument{\setlength\belowdisplayskip{4pt}}


%% Typography
\renewcommand*{\bibfont}{\small}
\ifbeamer{
    \AtBeginBibliography{\fontsize{3}{3}}
}{}


%% Table
\def\arraystretch{1.2}
\AtBeginDocument{\setlength{\tabcolsep}{0.5em}}

%% Default hyper setup
\ifsubmission
    \hypersetup{
        linktoc=all,
        pdfauthor={},
        pdftitle={Anonymous submission},
        pdfsubject={},
        pdfkeywords={}
    }
\fi
\setcounter{tocdepth}{2}


%% Cross-referencing: hyper-xr
\makeatletter
\newcommand*{\addFileDependency}[1]{% argument=file name and extension
  \typeout{(#1)}
  \@addtofilelist{#1}
  \IfFileExists{#1}{}{\typeout{No file #1.}}
}
\makeatother

\newcommand*{\myexternaldocument}[1]{%
    \externaldocument{#1}%
    \addFileDependency{#1.tex}%
    \addFileDependency{#1.aux}%
}


%% My TOC
\newcommand{\mytoc}{
    \begingroup
        \let\clearpage\relax
        \let\chapter\section
        \def\contentsname{Contents}
        \tableofcontents
    \endgroup
}


%% My maketitle
\newcommand{\mymaketitle}{
    \begingroup
        \let\oldaddcontentsline\addcontentsline
        \renewcommand*{\addcontentsline}[3]{%
            \ifstrequal{##1}{toc}{%
                \ifstrequal{##2}{author}{}{%
                    \ifstrequal{##2}{title}{}{%
                        \oldaddcontentsline{##1}{##2}{##3}
                    }
                }
            }{%
                \oldaddcontentsline{##1}{##2}{##3}
            }
        }
        %\let\addtocontents\relax
        \maketitle
    \endgroup
}


%% Auto delimiters
\let\ifdisplay\iffalse
\everydisplay\expandafter{\let\ifdisplay\iftrue}
\declarecommand{\DeclareAutoPairedDelimiter}[3]{%
    \declarecommand{#1}[1]{\ifdisplay\left#2\else#2\fi##1\ifdisplay\right#3\else#3\fi}
}


%% auto cref
\let\originalcref\cref
\renewcommand{\cref}[1]{\ifdisplay\textup{\originalcref{#1}}\else\originalcref{#1}\fi}


%% strikeout in math mode
\let\originalsout\sout
\renewcommand{\sout}[1]{\ifmmode\text{\originalsout{\ensuremath{#1}}}\else\originalsout{#1}\fi}