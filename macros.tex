% \newcommand{\wrap}[1]{\ensuremath{{#1}}\xspace}
\newcommand{\AlgoFont}[1]{\ensuremath{\mathcal{#1}}}
\newcommand{\ClassFont}[1]{\ensuremath{\mathrm{#1}}}
\newcommand{\FunctionFont}[1]{\ensuremath{\mathrm{#1}}}
\newcommand{\PartyFont}[1]{\ensuremath{\mathsf{#1}}}
\newcommand{\ProtocolFont}[1]{\ensuremath{\mathsf{#1}}}
\newcommand{\AdvFont}[1]{\ensuremath{\mathcal{#1}}}
\newcommand{\CodeFont}[1]{\ensuremath{\mathtt{#1}}}
\newcommand{\SetFont}[1]{\ensuremath{\mathit{#1}}}

% Generic
\newcommand{\TodoSuper}[2][]{\todo[inline, color=green!80, #1]{#2}}
\newcommand{\romanNumbering}[1]{\textsuperscript{(\romannum{#1})}}

% Parenthesis
\renewcommand{\left}{\mleft}
\renewcommand{\right}{\mright}
% https://tex.stackexchange.com/questions/225562/declarepaireddelimiter-with-left-and-right
\newcommand{\DeclareAutoPairedDelimiter}[3]{
  \expandafter\DeclarePairedDelimiter\csname Auto\string#1\endcsname{#2}{#3}%
  \begingroup\edef\x{\endgroup
    \noexpand\DeclareRobustCommand{\noexpand#1}{%
      \expandafter\noexpand\csname Auto\string#1\endcsname*}}%
  \x
}
\DeclarePairedDelimiter{\floor}{\lfloor}{\rfloor}
\DeclarePairedDelimiter{\ceil}{\lceil}{\rceil}
\DeclarePairedDelimiter{\round}{\lfloor}{\rceil}
\DeclarePairedDelimiter{\parr}{(}{)}
\DeclarePairedDelimiter{\pars}{[}{]}
\DeclarePairedDelimiter{\parc}{\{}{\}}
\DeclarePairedDelimiter{\abs}{\vert}{\vert}
\DeclarePairedDelimiter{\card}{\vert}{\vert}
\DeclarePairedDelimiter{\length}{\vert}{\vert}
\DeclarePairedDelimiter{\inprod}{\langle}{\rangle}
\DeclarePairedDelimiter{\parity}{\langle}{\rangle}
\DeclarePairedDelimiter{\at}{.}{\vert}
% https://tex.stackexchange.com/questions/45713/creating-a-large-such-that-symbol
\newcommand{\given}{\;\ifnum\currentgrouptype=16 \middle\fi\vert\;}

% Asymptotics
\DeclareMathOperator{\negl}{negl}
\DeclareMathOperator{\owhl}{owhl}
\DeclareMathOperator{\poly}{poly}
\DeclareMathOperator{\polylog}{polylog}
% \DeclareMathOperator{\superpoly}{superpoly}
\DeclareMathOperator{\strictlyFaster}{\omega}
\DeclareMathOperator{\faster}{\Omega}
\DeclareMathOperator{\equallyFast}{\Theta}
\DeclareMathOperator{\slower}{\mathcal{O}}
\DeclareMathOperator{\strictlySlower}{\mathrm{o}}

% Parameters & Notation
\newcommand{\n}{n}
%_\newcommand{\secpar}{{\glssymbol{secpar}}}
\newcommand{\barsecpar}{\hat{\secpar}}

% Sets
\newcommand{\bit}{\parc*{\false,\true}}
\newcommand{\bits}{\bit^*}
\newcommand{\secparbits}[1][]{\bit^{#1\secpar}}
% \newcommand{\barsecparbits}[1][]{\bit^{#1 \barsecpar}}
% \newcommand{\bool}{\parc*{\false,\true}}
\newcommand{\true}{1}
\newcommand{\false}{0}
\newcommand{\emptyString}{\varepsilon}
\newcommand{\mulgroup}[1]{\Z_{#1}^\times}
\newcommand{\Z}{\mathbb{Z}}
\newcommand{\R}{\mathbb{R}}
\newcommand{\N}{\mathbb{N}}

% Complexity Classes
\renewcommand{\P}{\AlgoFont{P}}
\newcommand{\NP}{\AlgoFont{NP}}
\newcommand{\DTIME}{\AlgoFont{DTIME}}
\newcommand{\NC}[1][]{\ClassFont{NC}^{#1}}

% Operations
% \newcommand{\BinDist}[1]{\operatorname{Bin}\parr*{#1}}
% \renewcommand{\mod}{\operatorname{mod}}
% \newcommand{\concat}{\mathopen{}\operatorname{||}\mathclose{}}
% \renewcommand{\d}{\mathrm{d}}
\newcommand{\xor}{\oplus}
\newcommand{\BigXOR}{\bigoplus}
\newcommand{\cupdot}{\mathbin{\mathaccent\cdot\cup}}
% \newcommand{\bigcupdot}{\mathop{\cdot\bigcup}}
\newcommand{\bigcupdot}{\mathop{\hspace{8.8pt}\cdot \hspace{-8.8pt}\bigcup\hspace{5pt}}}
% \DeclareMathOperator{\bigcupdot}{\hspace{8.8pt}\cdot \hspace{-8.8pt}\bigcup\limits}

% Relations
\newcommand{\isequal}{\overset{?}{=}}
% \newcommand{\implies}{\Rightarrow}
\newcommand{\nimplies}{\centernot\implies}
% \newcommand{\cind}{\overset{c}{\approx}}
\newcommand{\sind}{\overset{s}{\approx}}
%\newcommand{\UCRealize}{\overset{\mathrm{UC}}{\geq}}
\newcommand{\isomorphic}{\cong}
\newcommand{\pind}{\protect\mathpalette{\protect\independenT}{\perp}}
\def\independenT#1#2{\mathrel{\rlap{$#1#2$}\mkern2mu{#1#2}}}

% Randomness
\renewcommand{\Pr}[2][]{\operatorname{Pr}_{#1}\pars*{#2}}
\newcommand{\Ex}[1]{\operatorname{E}\pars*{#1}}
\newcommand{\Var}[1]{\operatorname{Var}\pars*{#1}}
\newcommand{\getsr}{\overset{\$}{\gets}}
% \newcommand{\uniform}[1]{\mathcal{U}\parr*{#1}}


%% Conditional variables, since our submission has to be anonymous.
\newcommand{\ifcameraready}{\iftrue}
\newcommand{\ifeprint}{\iftrue}
\newcommand{\ifcgconstruction}{\iftrue}
\newcommand{\ifcapitalize}{\iftrue}
\newcommand{\ifframedfunctionality}{\iftrue}