% \providecommand{\wrap}[1]{\ensuremath{{#1}}\xspace}
\providecommand{\AlgoFont}[1]{\ensuremath{\mathcal{#1}}}
\providecommand{\ClassFont}[1]{\ensuremath{\mathrm{#1}}}
\providecommand{\FunctionFont}[1]{\ensuremath{\mathrm{#1}}}
\providecommand{\PartyFont}[1]{\ensuremath{\mathsf{#1}}}
\providecommand{\ProtocolFont}[1]{\ensuremath{\mathsf{#1}}}
\providecommand{\AdvFont}[1]{\ensuremath{\mathcal{#1}}}
\providecommand{\CodeFont}[1]{\ensuremath{\mathtt{#1}}}
\providecommand{\SetFont}[1]{\ensuremath{\mathit{#1}}}

% Generic
\providecommand{\TodoSuper}[2][]{\todo[inline, color=green!80, #1]{#2}}
\providecommand{\romanNumbering}[1]{\textsuperscript{(\romannum{#1})}}

% Parenthesis
\providecommand{\left}{\mleft}
\providecommand{\right}{\mright}
\DeclareAutoPairedDelimiter{\floor}{\lfloor}{\rfloor}
\DeclareAutoPairedDelimiter{\ceil}{\lceil}{\rceil}
\DeclareAutoPairedDelimiter{\round}{\lfloor}{\rceil}
\DeclareAutoPairedDelimiter{\parr}{(}{)}
\DeclareAutoPairedDelimiter{\pars}{[}{]}
\DeclareAutoPairedDelimiter{\parc}{\{}{\}}
\DeclareAutoPairedDelimiter{\abs}{\vert}{\vert}
\DeclareAutoPairedDelimiter{\card}{\vert}{\vert}
\DeclareAutoPairedDelimiter{\length}{\vert}{\vert}
\DeclareAutoPairedDelimiter{\inprod}{\langle}{\rangle}
\DeclareAutoPairedDelimiter{\parity}{\langle}{\rangle}
\DeclareAutoPairedDelimiter{\at}{.}{\vert}
% https://tex.stackexchange.com/questions/45713/creating-a-large-such-that-symbol
\providecommand{\given}{\;\ifnum\currentgrouptype=16 \middle\fi\vert\;}

% Asymptotics
\DeclareMathOperator{\negl}{negl}
\DeclareMathOperator{\owhl}{owhl}
\DeclareMathOperator{\poly}{poly}
\DeclareMathOperator{\polylog}{polylog}
% \DeclareMathOperator{\superpoly}{superpoly}
\DeclareMathOperator{\strictlyFaster}{\omega}
\DeclareMathOperator{\faster}{\Omega}
\DeclareMathOperator{\equallyFast}{\Theta}
\DeclareMathOperator{\slower}{\mathcal{O}}
\DeclareMathOperator{\strictlySlower}{\mathrm{o}}

% Parameters & Notation
\providecommand{\n}{n}
%_\providecommand{\secpar}{{\glssymbol{secpar}}}
\providecommand{\barsecpar}{\hat{\secpar}}

% Sets
\providecommand{\bit}{\parc*{\false,\true}}
\providecommand{\bits}{\bit^*}
\providecommand{\secparbits}[1][]{\bit^{#1\secpar}}
% \providecommand{\barsecparbits}[1][]{\bit^{#1 \barsecpar}}
% \providecommand{\bool}{\parc*{\false,\true}}
\providecommand{\true}{1}
\providecommand{\false}{0}
\providecommand{\emptyString}{\varepsilon}
\providecommand{\mulgroup}[1]{\Z_{#1}^\times}
\providecommand{\Z}{\mathbb{Z}}
\providecommand{\R}{\mathbb{R}}
\providecommand{\N}{\mathbb{N}}

% Complexity Classes
\providecommand{\P}{\AlgoFont{P}}
\providecommand{\NP}{\AlgoFont{NP}}
\providecommand{\DTIME}{\AlgoFont{DTIME}}
\providecommand{\NC}[1][]{\ClassFont{NC}^{#1}}

% Operations
% \providecommand{\BinDist}[1]{\operatorname{Bin}\parr*{#1}}
% \providecommand{\mod}{\operatorname{mod}}
% \providecommand{\concat}{\mathopen{}\operatorname{||}\mathclose{}}
% \providecommand{\d}{\mathrm{d}}
\providecommand{\xor}{\oplus}
\providecommand{\BigXOR}{\bigoplus}
\providecommand{\cupdot}{\mathbin{\mathaccent\cdot\cup}}
% \providecommand{\bigcupdot}{\mathop{\cdot\bigcup}}
\providecommand{\bigcupdot}{\mathop{\hspace{8.8pt}\cdot \hspace{-8.8pt}\bigcup\hspace{5pt}}}
% \DeclareMathOperator{\bigcupdot}{\hspace{8.8pt}\cdot \hspace{-8.8pt}\bigcup\limits}

% Relations
\providecommand{\isequal}{\overset{?}{=}}
% \providecommand{\implies}{\Rightarrow}
\providecommand{\nimplies}{\centernot\implies}
% \providecommand{\cind}{\overset{c}{\approx}}
\providecommand{\sind}{\overset{s}{\approx}}
%\providecommand{\UCRealize}{\overset{\mathrm{UC}}{\geq}}
\providecommand{\isomorphic}{\cong}
\providecommand{\pind}{\protect\mathpalette{\protect\independenT}{\perp}}
\def\independenT#1#2{\mathrel{\rlap{$#1#2$}\mkern2mu{#1#2}}}

% Randomness
\providecommand{\Pr}[2][]{\operatorname{Pr}_{#1}\pars*{#2}}
\providecommand{\Ex}[1]{\operatorname{E}\pars*{#1}}
\providecommand{\Var}[1]{\operatorname{Var}\pars*{#1}}
\providecommand{\getsr}{\overset{\$}{\gets}}
% \providecommand{\uniform}[1]{\mathcal{U}\parr*{#1}}