% \setcommand{\wrap}[1]{\ensuremath{{#1}}\xspace}
\setcommand{\AlgoFont}[1]{\ensuremath{\mathcal{#1}}}
\setcommand{\ClassFont}[1]{\ensuremath{\mathrm{#1}}}
\setcommand{\FunctionFont}[1]{\ensuremath{\mathrm{#1}}}
\setcommand{\PartyFont}[1]{\ensuremath{\mathsf{#1}}}
\setcommand{\ProtocolFont}[1]{\ensuremath{\mathsf{#1}}}
\setcommand{\AdvFont}[1]{\ensuremath{\mathcal{#1}}}
\setcommand{\CodeFont}[1]{\ensuremath{\mathtt{#1}}}
\setcommand{\SetFont}[1]{\ensuremath{\mathit{#1}}}

% Generic
\setcommand{\TodoSuper}[2][]{\todo[inline, color=green!80, #1]{#2}}
\setcommand{\romanNumbering}[1]{\textsuperscript{(\romannum{#1})}}

% Parenthesis
\setcommand{\left}{\mleft}
\setcommand{\right}{\mright}
\DeclareAutoPairedDelimiter{\floor}{\lfloor}{\rfloor}
\DeclareAutoPairedDelimiter{\ceil}{\lceil}{\rceil}
\DeclareAutoPairedDelimiter{\round}{\lfloor}{\rceil}
\DeclareAutoPairedDelimiter{\parr}{(}{)}
\DeclareAutoPairedDelimiter{\pars}{[}{]}
\DeclareAutoPairedDelimiter{\parc}{\{}{\}}
\DeclareAutoPairedDelimiter{\abs}{\vert}{\vert}
\DeclareAutoPairedDelimiter{\card}{\vert}{\vert}
\DeclareAutoPairedDelimiter{\length}{\vert}{\vert}
\DeclareAutoPairedDelimiter{\bracket}{\langle}{\rangle}
\DeclareAutoPairedDelimiter{\inprod}{\langle}{\rangle}
\DeclareAutoPairedDelimiter{\parity}{\langle}{\rangle}
\DeclareAutoPairedDelimiter{\at}{\ifdisplay.\else{}\fi}{\vert}
% https://tex.stackexchange.com/questions/45713/creating-a-large-such-that-symbol
\setcommand{\given}{\;\ifnum\currentgrouptype=16 \middle\fi\vert\;}
% https://tex.stackexchange.com/questions/182749/underbrace-matrix-inside-brackets
\setcommand{\smashedunderbrace}[2]{\smash{\underbrace{#1}_{#2}}\vphantom{#1}}
\setcommand{\smashedoverbrace}[2]{\smash{\overbrace{#1}^{#2}}\vphantom{#1}}

% Asymptotics
\DeclareMathOperator{\negl}{negl}
\DeclareMathOperator{\owhl}{owhl}
\DeclareMathOperator{\poly}{poly}
\DeclareMathOperator{\polylog}{polylog}
% \DeclareMathOperator{\superpoly}{superpoly}
\DeclareMathOperator{\strictlyFaster}{\omega}
\DeclareMathOperator{\faster}{\Omega}
\DeclareMathOperator{\equallyFast}{\Theta}
\DeclareMathOperator{\slower}{\mathcal{O}}
\DeclareMathOperator{\strictlySlower}{\mathrm{o}}

% Parameters & Notation
\setcommand{\n}{n}
%_\setcommand{\secpar}{{\glssymbol{secpar}}}
\setcommand{\barsecpar}{\hat{\secpar}}

% Sets
\setcommand{\bit}{\parc{\false,\true}}
\setcommand{\bits}{\bit^*}
\setcommand{\secparbits}[1][]{\bit^{#1\secpar}}
\setcommand{\true}{1}
\setcommand{\false}{0}
\setcommand{\emptyString}{\varepsilon}
\setcommand{\mulgroup}[1]{\Z_{#1}^\times}
\setcommand{\Z}{\mathbb{Z}}
\setcommand{\R}{\mathbb{R}}
\setcommand{\N}{\mathbb{N}}

% Complexity Classes
\setcommand{\P}{\AlgoFont{P}}
\setcommand{\NP}{\AlgoFont{NP}}
\setcommand{\DTIME}{\AlgoFont{DTIME}}
\setcommand{\NC}[1][]{\ClassFont{NC}^{#1}}

% Operations
% \setcommand{\mod}{\operatorname{mod}}
% \setcommand{\concat}{\mathopen{}\operatorname{||}\mathclose{}}
% \setcommand{\d}{\mathrm{d}}
\setcommand{\xor}{\oplus}
\setcommand{\BigXOR}{\bigoplus}
\setcommand{\cupdot}{\mathbin{\mathaccent\cdot\cup}}
% \setcommand{\bigcupdot}{\mathop{\cdot\bigcup}}
\setcommand{\bigcupdot}{\mathop{\hspace{8.8pt}\cdot \hspace{-8.8pt}\bigcup\hspace{5pt}}}
% \DeclareMathOperator{\bigcupdot}{\hspace{8.8pt}\cdot \hspace{-8.8pt}\bigcup\limits}
\DeclareMathOperator*{\argmin}{argmin}
\DeclareMathOperator*{\argmax}{argmax}

% Relations
\setcommand{\isequal}{\overset{?}{=}}
% \setcommand{\implies}{\Rightarrow}
\setcommand{\nimplies}{\centernot\implies}
% \setcommand{\cind}{\overset{c}{\approx}}
\setcommand{\sind}{\overset{s}{\approx}}
%\setcommand{\UCRealize}{\overset{\mathrm{UC}}{\geq}}
\setcommand{\isomorphic}{\cong}
\setcommand{\pind}{\protect\mathpalette{\protect\independenT}{\perp}}
\def\independenT#1#2{\mathrel{\rlap{\(#1#2\)}\mkern2mu{#1#2}}}
\let\originalleq\leq
\setcommand{\leq}[1][]{\originalleq_{\mathsf{#1}}}
\setcommand{\l}[1][]{<_{\mathsf{#1}}}
\let\originalgeq\geq
\setcommand{\geq}[1][]{\originalgeq_{\mathsf{#1}}}
\setcommand{\g}[1][]{>_{\mathsf{#1}}}


% Randomness
\setcommand{\Pr}[2][]{\operatorname{Pr}_{#1}\pars{#2}}
\setcommand{\Ex}[2][]{\operatorname{Ex}_{#1}\pars{#2}}
\setcommand{\Var}[2][]{\operatorname{Var}_{#1}\pars{#2}}
\setcommand{\getsr}{\overset{\$}{\gets}}

% Symbols
\setcommand{\ihat}{{\hat{\iota}}}
\setcommand{\jhat}{{\hat{\jmath}}}