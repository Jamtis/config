% \declarecommand{\wrap}[1]{\ensuremath{{#1}}\xspace}
\declarecommand{\AlgoFont}[1]{\ensuremath{\mathcal{#1}}}
\declarecommand{\ClassFont}[1]{\ensuremath{\mathrm{#1}}}
\declarecommand{\FunctionFont}[1]{\ensuremath{\mathrm{#1}}}
\declarecommand{\PartyFont}[1]{\ensuremath{\mathsf{#1}}}
\declarecommand{\ProtocolFont}[1]{\ensuremath{\mathsf{#1}}}
\declarecommand{\AdvFont}[1]{\ensuremath{\mathcal{#1}}}
\declarecommand{\CodeFont}[1]{\ensuremath{\mathtt{#1}}}
\declarecommand{\SetFont}[1]{\ensuremath{\mathit{#1}}}

% Generic
\declarecommand{\TodoSuper}[2][]{\todo[inline, color=green!80, #1]{#2}}
\declarecommand{\romanNumbering}[1]{\textsuperscript{(\romannum{#1})}}

% Parenthesis
\declarecommand{\left}{\mleft}
\declarecommand{\right}{\mright}
\DeclareAutoPairedDelimiter{\floor}{\lfloor}{\rfloor}
\DeclareAutoPairedDelimiter{\ceil}{\lceil}{\rceil}
\DeclareAutoPairedDelimiter{\round}{\lfloor}{\rceil}
\DeclareAutoPairedDelimiter{\parr}{(}{)}
\DeclareAutoPairedDelimiter{\pars}{[}{]}
\DeclareAutoPairedDelimiter{\parc}{\{}{\}}
\DeclareAutoPairedDelimiter{\abs}{\vert}{\vert}
\DeclareAutoPairedDelimiter{\card}{\vert}{\vert}
\DeclareAutoPairedDelimiter{\length}{\vert}{\vert}
\DeclareAutoPairedDelimiter{\inprod}{\langle}{\rangle}
\DeclareAutoPairedDelimiter{\parity}{\langle}{\rangle}
\DeclareAutoPairedDelimiter{\at}{.}{\vert}
% https://tex.stackexchange.com/questions/45713/creating-a-large-such-that-symbol
\declarecommand{\given}{\;\ifnum\currentgrouptype=16 \middle\fi\vert\;}
% https://tex.stackexchange.com/questions/182749/underbrace-matrix-inside-brackets
\newcommand{\smashedunderbrace}[2]{\smash{\underbrace{#1}_{#2}}\vphantom{#1}}
\newcommand{\smashedoverbrace}[2]{\smash{\overbrace{#1}_{#2}}\vphantom{#1}}

% Asymptotics
\DeclareMathOperator{\negl}{negl}
\DeclareMathOperator{\owhl}{owhl}
\DeclareMathOperator{\poly}{poly}
\DeclareMathOperator{\polylog}{polylog}
% \DeclareMathOperator{\superpoly}{superpoly}
\DeclareMathOperator{\strictlyFaster}{\omega}
\DeclareMathOperator{\faster}{\Omega}
\DeclareMathOperator{\equallyFast}{\Theta}
\DeclareMathOperator{\slower}{\mathcal{O}}
\DeclareMathOperator{\strictlySlower}{\mathrm{o}}

% Parameters & Notation
\declarecommand{\n}{n}
%_\declarecommand{\secpar}{{\glssymbol{secpar}}}
\declarecommand{\barsecpar}{\hat{\secpar}}

% Sets
\declarecommand{\bit}{\parc{\false,\true}}
\declarecommand{\bits}{\bit^*}
\declarecommand{\secparbits}[1][]{\bit^{#1\secpar}}
\declarecommand{\true}{1}
\declarecommand{\false}{0}
\declarecommand{\emptyString}{\varepsilon}
\declarecommand{\mulgroup}[1]{\Z_{#1}^\times}
\declarecommand{\Z}{\mathbb{Z}}
\declarecommand{\R}{\mathbb{R}}
\declarecommand{\N}{\mathbb{N}}

% Complexity Classes
\declarecommand{\P}{\AlgoFont{P}}
\declarecommand{\NP}{\AlgoFont{NP}}
\declarecommand{\DTIME}{\AlgoFont{DTIME}}
\declarecommand{\NC}[1][]{\ClassFont{NC}^{#1}}

% Operations
% \declarecommand{\mod}{\operatorname{mod}}
% \declarecommand{\concat}{\mathopen{}\operatorname{||}\mathclose{}}
% \declarecommand{\d}{\mathrm{d}}
\declarecommand{\xor}{\oplus}
\declarecommand{\BigXOR}{\bigoplus}
\declarecommand{\cupdot}{\mathbin{\mathaccent\cdot\cup}}
% \declarecommand{\bigcupdot}{\mathop{\cdot\bigcup}}
\declarecommand{\bigcupdot}{\mathop{\hspace{8.8pt}\cdot \hspace{-8.8pt}\bigcup\hspace{5pt}}}
% \DeclareMathOperator{\bigcupdot}{\hspace{8.8pt}\cdot \hspace{-8.8pt}\bigcup\limits}

% Relations
\declarecommand{\isequal}{\overset{?}{=}}
% \declarecommand{\implies}{\Rightarrow}
\declarecommand{\nimplies}{\centernot\implies}
% \declarecommand{\cind}{\overset{c}{\approx}}
\declarecommand{\sind}{\overset{s}{\approx}}
%\declarecommand{\UCRealize}{\overset{\mathrm{UC}}{\geq}}
\declarecommand{\isomorphic}{\cong}
\declarecommand{\pind}{\protect\mathpalette{\protect\independenT}{\perp}}
\def\independenT#1#2{\mathrel{\rlap{\(#1#2\)}\mkern2mu{#1#2}}}

% Randomness
\declarecommand{\Pr}[2][]{\operatorname{Pr}_{#1}\pars{#2}}
\declarecommand{\Ex}[1]{\operatorname{E}\pars{#1}}
\declarecommand{\Var}[1]{\operatorname{Var}\pars{#1}}
\declarecommand{\getsr}{\overset{\$}{\gets}}